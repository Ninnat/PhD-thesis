\newpage
\begin{abstract}

\noindent A classical simulation scheme of quantum computation given a restricted set of states and measurements may be---occasionally, but only occasionally---interpreted naturally as a statistical simulation of positive quasi-probability distributions on a phase space. In this dissertation, we explore phase space representations for finite-dimensional quantum systems and their negativities beyond the usual analogues of the Wigner function.

The first line of study focuses on a characterization tool for valid quasi-probability distributions of (possibly mixed) quantum states. A quantum generalization of Bochner's theorem from classical probability theory simultaneously characterizes both the set of valid Wigner functions and the subset of positive ones. We extend this theorem to discrete phase spaces based on projective representations of abelian groups, including the discrete Wigner functions of Gross (J. Math. Phys. 47, 122107 (2006)) and Gibbons \emph{et al.} (Phys. Rev. A 70, 062101 (2004)).

The rest of the dissertation is then dedicated to phase space representations covariant with respect to a general Lie group. This means, in particular, that operators that lie in the same orbit of the group have the same amount of negativity in their quasi-probabilities---a natural requirement if one would like to interpret negativity as non-classicality, and the group action as a classical dynamics. We show that the construction due to Brif and Mann (Phys. Rev. A 59, 971 (1999)) of group covariant phase space representations satisfying the so-called Stratonovich-Weyl correspondence is essentially unique, given the group and an input state in the relevant representation. As an application, we construct quasi-probability representations on a compact phase space of fermionic Gaussian states tailored to a classically simulatable problem of fermionic linear optics and find that, in the first non-trivial case of four fermionic modes, the quasi-probabilities of the Gaussian states exhibit a considerable amount of negativity.

%\cmc{I think the style in abstracts is to put in references explicitly, in the most abbreviated form, since the abstract might appear separately from the rest of the document.}
\end{abstract}

