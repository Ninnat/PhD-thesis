We provide here a complete proof of the discrete Hudson's theorem, following a sketch in an unpublished talk by Gross \cite{gross2015coogee}.\footnote{I am grateful to Marcus Appleby for sharing his extensive knowledge on discrete Wigner functions and pointing out the group property of the frame for Gross's Wigner function.}

Hudson's theorem states that the Wigner function of a pure state is positive if and only if it is Gaussian \cite{hudson_when_1974}. The theorem in the multi-mode scenario and other generalizations was given in \cite{soto_when_1983,toft_hudsons_2006} and references therein. In the discrete case we have

\begin{theorem}{\normalfont (Discrete Hudson's theorem)}
	Gross' Wigner function \eqref{ch4:gross-wigner-function} of a pure state $\ket{\psi}$ is positive if and only if $\ket{\psi}$ is a stabilizer state
	{\normalfont \cite{gross_hudsons_2006,gross_non-negative_2007}.}
\end{theorem}

\emph{Stabilizer states} were originally defined for multiple qubits \cite{gottesman_stabilizer_1997} and have a plethora of applications in quantum error correction and quantum computation \cite{fujii2015quantum}.
In odd dimensions, they have the following equivalent characterizations:
\begin{enumerate}
	
	\item\label{def-stabilizer} Each stabilizer state is a joint eigenvector of a \emph{stabilizer subgroup} -- a maximal abelian subgroup of the generalized Pauli group $\mathcal{P}_d$.
	
	\item\label{clifford-orbit} Stabilizer states are the orbit of the state $\ket{0}$ under the Clifford group.
	
	\item\label{gaussian-wave-function} Their wave functions have a Gaussian form in the standard basis $\{\ket{m}\mid m=0,\dots, d-1 \}$
	\begin{align}
	\psi(m) = \omega^{mAm + bm},
	\end{align}
	where $A$ is a symmetric matrix with entries in $\mathbb{Z}_d$ and $b \in \mathbb{Z}_d$. %{\cmc Is there a difference between the sub and superscript $d$?}
	
\end{enumerate}
Properties \ref{clifford-orbit} and \ref{gaussian-wave-function} especially affirm the status of stabilizer states as the natural discrete analogue of Gaussian states.

To prove the discrete Hudson's theorem in one direction, one computes the discrete Wigner function of all stabilizer states directly and observes that they are positive. In fact, the support of a stabilizer state always lies on a line (as defined in finite geometry) \cite{gross_hudsons_2006}, so they are more like infinitely squeezed states.

A short proof in the other direction given in an unpublished talk by Gross \cite{gross2015coogee} can be summarized as follows: a positive discrete Wigner function of
a pure state simultaneously maximizes the 2-norm and minimizes the 1-norm, forcing it to be a stabilizer state.
\begin{lemma}
	\label{lem:supp_size}A positive Wigner function $\mu_{\psi}$ of
	a pure state $\rho= \ketbra{\psi}{\psi}$ must be constant on
	its support of size $d$.
\end{lemma}
\begin{proof}
	We want to bound the 2-norm of $\rho$,
	\begin{align}
	\norm \rho_{2}^{2} & = \Tr (\rho^{\dagger}\rho)
	= \sum_{j,k} |\rho_{jk}|^{2}.
	\end{align}
	Every displacement operator except $\id$ has zero trace, so the trace of $A(0,0)$ and thus every phase point operator is 1. Moreover, $ \{ A({\bf u})/\sqrt{d} \} $ is an orthonormal basis because $A({\bf u})$ squares to the identity. Therefore,
	\begin{align}
	\frac{1}{d} \le  \sum_{j,k} |\rho_{jk}|^{2} &= \frac{1}{d} \sum_{{\bf u}} \left| \Tr [A({\bf u}) \rho] \right|^2 \le 1.
	\end{align}
	$\rho$ is a pure state if and only if $ \norm \rho _{2}^{2} = \Tr (\rho^2) = 1$, which means that on the one hand,
	\begin{align}
	\sum_{{\bf u}} |\braket{\psi|A({\bf u})|\psi}|^2 &= d, \label{eq:2-norm}
	\end{align}
	i.e. the 2-norm is maximized. On the other hand,
	\begin{align}
	\sum_{{\bf u}} \braket{\psi|A({\bf u})|\psi} &= d.
	\end{align}
	Every term in the sum is positive if and only if its 1-norm is minimized
	\begin{align}
	\sum_{{\bf u}} |\braket{\psi|A({\bf u})|\psi}| &= d.\label{eq:1-norm}
	\end{align}
	Each term in the sum can be bounded by the sum of the product of the singular values using von Neumann's trace inequality:
	\begin{align}
	|\Tr(AB)| &\le \sum_j a_j b_j,\label{eq:trace_ineq}
	\end{align}
	where $a_{j}$ and $b_{j}$ are singular values of $A$ and $B$ respectively,
	ordered by their magnitudes. The bound is saturated if and only if $A$ and
	$B$ are normal and commute. Since $\rho$ is a rank-one projection
	operator, we only need the largest singular value of $A({\bf u})$,
	which is 1,
	\begin{align}
	|\braket{\psi|A({\bf u})|\psi}| &\le \norm{A({\bf u})}_{\infty} = 1\label{eq:baby_holder_ineq}
	\end{align}
	But if this bound is not saturated, the 2-norm will not reach the
	maximum. To saturate the bound, $\mu_{\rho}$ can have a support only
	on $d$ points and $\ket{\psi}$ must be a joint eigenvector of $d$
	phase point operators.
\end{proof}	

\noindent In particular, $\ket{\psi}$ is an eigenvector of every product of these $d$ phase point operators, which makes it also an eigenvector of some displacement operators because
\begin{align}\label{phase-point-clifford}
	A({\bf u}) A({\bf v})
		&= \left( D\dgg ({\bf u}) A({\bf 0}) D ({\bf u}) \right)
			\left( D\dgg ({\bf v}) A({\bf 0}) D ({\bf v}) \right) \nonumber \\
		&= \omega^{-[{\bf u,v}]/2} D({\bf -u})	\left[ A({\bf 0}) D ({\bf u-v}) A({\bf 0}) \right] D ({\bf v}) \nonumber \\
		&= \omega^{-[{\bf u,v}]/2} D({\bf -u})D ({\bf v-u}) D ({\bf v}) \nonumber \\
		&= \omega^{-2 [{\bf u,v}]} D(2{\bf v} - 2{\bf u}).
\end{align}
When ${\bf w = v-u} \neq 0$, $2{\bf w}$ is never zero since $d$ is odd. So for a given ${\bf w}$, $\ket{\psi}$ is an eigenvector of $d$ commuting displacement operators $\{ \id, D({\bf w}), \dots, D^{d-1}({\bf w}) \}$ and there can be no more than $d$ commuting displacement operators since $D({\bf w})$ has period $d$. $\ket{\psi}$ is therefore a stabilizer state and the discrete Hudson's theorem is proved. 