\newcommand\rep{U}
\setlength\epigraphwidth{9.25cm}
\epigraph{Quantum kinematics [is] an abelian group of rotations.}{--- Hermann Weyl}

%------------------------------------
\section{Overview}
%------------------------------------

The opening quote is a modification of the chapter title ``Quantum kinematics as an abelian group of rotations" from \cite{weyl1950qm}. What Weyl meant was that, even though the symmetry group generated by translations in position and momentum do not commute, they do commute \emph{up to a phase}. In other words, quantum kinematics of $n$ particles is a \emph{projective} representation of the abelian group $\mathbb{R}^{2n}$, which can be identified with the classical phase space. We shall call a phase space that can be identified with a commutative group a \emph{commutative phase space}.

The most studied quasi-probability representations, namely the Wigner, P, and Q functions, are defined on the commutative phase space $\mathbb{R}^{2n}$. A wealth of excellent reviews already exists \cite{hillery_distribution_1984, lee_theory_1995}, so we will give just a brief discussion in \autoref{ch4:wigner} with an eye toward generalizing the Wigner function to finite-dimensional Hilbert spaces. In the next section, \autoref{ch4:discrete-wigner}, we introduce the closest discrete analogue of the Wigner function \cite{gross_hudsons_2006,gross_non-negative_2007} and discuss why it can only be defined in odd dimensions. The discrete Hudson's theorem which characterizes the set of positive pure states in Gross's Wigner function is relegated to Appendix \ref{app:hudson}. The quantum Bochner's theorem, which characterizes the set of (possibly mixed) positive states of the continuous Wigner function, is introduced in \autoref{ch4:bochner}. The chapter contains no new result up to this point; then \autoref{ch4:discrete-bochner} is an original result published in \cite{dangniam_quantum_2015} that generalizes the quantum Bochner's theorem to discrete phase spaces.  %The theorem applies to the Wigner functions of Gross, Gibbons \cite{gibbons_discrete_2004}, Leonhardt \cite{leonhardt_quantum-state_1995,leonhardt_discrete_1996}, and other unnamed commutative quasi-probability representations.

%------------------------------------
\section{The Wigner function}\label{ch4:wigner}
%------------------------------------

The Wigner representation provides an independent formulation of quantum theory that most resembles the classical symplectic phase space picture. (We will examine the ``independent" aspect more closely in \autoref{ch4:bochner} on the quantum Bochner's theorem.)
The central relation in quantum theory analogous to the Poisson bracket
\begin{align}\label{canonical-Poisson-bracket}
\{ q,p \} = 1
\end{align}
is the canonical commutation relation
\begin{align}\label{CCR}
[Q,P] &= i.
\end{align}
We would like a joint probability distribution $\mu_{\rho}(p,q)$ of the values of $Q$ and $P$. From the postulates of quantum theory we have a rule for calculating expectation values. The \emph{characteristic function} of a joint probability distribution $\mu(q,p)$ on the classical phase space $\mathbb{R}^2$ is the Fourier transform
\begin{align}\label{characteristic-function}
\phi(\zeta,\eta) = \frac{1}{2\pi} \iint \mu(p,q) e^{-i(\zeta q -\eta p)}\,dq\,dp.
\end{align}
(The opposite sign for $q$ and $p$ in the Fourier coefficient is chosen to reflect the symplectic structure of a classical phase space.)  Noting that the characteristic function is the expectation value of $e^{-i(\zeta q-\eta p)}$, we quantize $q$ and $p$ to obtain
\begin{align}\label{quantum-characteristic-function}
\phi_\rho(\zeta,\eta) =  \frac{1}{2\pi} \av{e^{ -i(\zeta Q-\eta P)}} = \frac{1}{2\pi}\Tr( e^{ -i(\zeta Q-\eta P)} \rho).
\end{align}
Since the characteristic function is the Fourier transform of the joint probability distribution, eq. \ref{quantum-characteristic-function} can be inverted, resulting in
\begin{align}\label{def-wigner'}
\mu_\rho(p,q)=\frac{1}{(2\pi)^2}\iint \Tr(e^{-i(\zeta Q-\eta P)}\rho) e^{i (\zeta q -\eta p)}\,d\zeta\,d\eta,
\end{align}
which is nothing but the Wigner function of $\rho$ \cite{weyl_quantenmechanik_1927, wigner_quantum_1932, groenewold_principles_1946, moyal_quantum_1949}. (Many of the classic references are reprinted in \cite{zachos2005quantum}.) It can also be written in the form of the \emph{Wigner map}
\begin{align}
\mu_{\rho}(q,p) &= \frac{1}{2\pi} \int e^{-ipy} \bigg\langle q+\frac{y}{2} \bigg| \rho \bigg|q-\frac{y}{2}\bigg\rangle dy.
\end{align}
The P and Q functions share the same phase space as the Wigner function and can be obtained from the Wigner function by convolution with a Gaussian $\exp[\pm (\xi^2 + \eta^2)/4]$.
%s-ordering

Both eq. \eqref{canonical-Poisson-bracket} and eq. \eqref{CCR} are representations of the Lie algebra of the \emph{Weyl-Heisenberg group}. %$\gr{H}{1,\mathbb{R}}$ where $i$ here is thought of as a generator of the center of the Lie algebra: $[i,Q] = [i,P] = 0$. In an explicit matrix form,
%\begin{align}
%\gr{H}{1,\mathbb{R}} = \left\{	\begin{pmatrix}
%		0 & a & c \\
%		0 & 0 & b \\
%		0 & 0& 0
%	\end{pmatrix}
%	\middle| a,b,c \in \mathbb{R} \right\},
%\end{align}
%where $a$ and $b$ are generated by $Q$ and $P$, and $c$ is generated by $i$. Its representations give projective representations of the classical phase space $\mathbb{R}^2$.
Physicists often write the group elements in the form of \emph{displacement operators}\footnote{In terms of bosonic creation and annihilation operators $D(\alpha) = e^{\alpha b^{\dagger} - \alpha^* b}$, where
\begin{align}
	b &= \frac{x + ip}{\sqrt{2}}, &
	x &= \frac{b+b^{\dagger}}{\sqrt{2}}, &
	p &= \frac{b-b^{\dagger}}{\sqrt{2}i},
\end{align}
and $\alpha = (q+ip)/\sqrt{2}$.
}
\begin{align}\label{displacement-operator}
D({\bf x}) = D(q,p) = e^{i(pQ - qP)}.
\end{align}
The Wigner function can be written as
\begin{align}
\mu_{\rho}(q,p) &= \frac{1}{2\pi} \Tr [A(q,p) \rho],
\end{align}
where
\begin{align}
A(q,p) &= \frac{1}{2\pi} \iint e^{i(q\eta - p\xi)} D^{\dagger}(\xi,\eta) d\xi d\eta
	= \frac{1}{2\pi} \iint e^{i[{\bf x}, {\bf y}]} D^{\dagger}({\bf y}) d^2{\bf y}.
\end{align}
is the \emph{phase point operator} ($A(q,p)/(2\pi)$ is the frame), and $[{\bf x}, {\bf y}]$ is the symplectic inner product: $[{\bf x}, {\bf y}] = q\eta - p\xi$ for ${\bf x} = (q,p)$ and ${\bf y} = (\xi,\eta)$. The phase point operator at the origin is the \emph{parity operator}:
\begin{align}
	A(0,0)QA(0,0) &= -Q, & A(0,0)PA(0,0) &= -P,
\end{align}
and all other phase point operators can be obtained from $A(0,0)$ by the displacement operators
\begin{align}
	A(q,p) &= D^{\dagger}(q,p) A(0,0) D(q,p),
\end{align}

%------------------------------------
\section{The discrete Wigner function}\label{ch4:discrete-wigner}
%------------------------------------

Using a special case of the Baker-Campbell-Haursdorff formula when $[A,B]$ commutes with $A$ and $B$:
\begin{align}
	e^{A+B} = e^A e^B e^{-[A,B]/2},
\end{align}
the displacement operator can be broken up as
\begin{align}
D(q,p) &= e^{-iqp/2} e^{ipQ} e^{-iqP} = e^{iqp/2} e^{-iqP} e^{ipQ}.
\end{align}
The actions of the individual exponential operators in the position basis $\{\ket{x}\}$ and the momentum basis $\ket{k} = \int e^{ikx} \ket{x} dx$ are
\begin{align}
e^{-iqP} \ket{x} &= \ket{x+q}, & e^{ipQ}\ket{x} &= e^{ipx}\ket{x}, \\
e^{ipQ}\ket{k} &= \ket{k+p}, & e^{-iqP} \ket{k} &= e^{-iqk}\ket{k}.
\end{align}

In the finite-dimensional Hilbert space $\hilb = \mathbb{C}^d$ with the standard basis $\{\ket{m}\mid m=0,\dots, d-1 \}$ and the Fourier basis $\{\ket{k} = d^{-1/2} \allowbreak \sum_{m=0}^{d-1} e^{2\pi ikm/d} \ket{m}\}$, define the \emph{generalized Pauli matrices} to be the shift and phase operators
\begin{align}
X\ket{m} &= \ket{m+1}, & Z\ket{m} &= \omega^m \ket{m}, \label{discrete-position-basis} \\
Z\ket{k} &= \ket{k+1}, & X\ket{k} &= \omega^{-k} \ket{k},
\end{align}
where $\omega = e^{2\pi i/d}$.
Thus, $X$ plays the role of $e^{-iP}$ and $Z$ plays the role of $e^{iQ}$ (but note that the phase space is now periodic). The qu\emph{d}it \emph{generalized Pauli group} $\mathcal{P}_d$ is the group generated by $X,Z$ and $\omega$ with the relation $ZX=\omega XZ$. The maximal period $N$ of an element in $\mathcal{P}_d$ is  $d$ in odd dimensions and $2d$ in even dimensions. (For example, the period of $XZ$ in 2 dimensions is 4.)  For now, let us define displacement operators
\begin{align}
	D({\bf u}) = D(x,z) &= \omega^{\varphi (x,z)} X^x Z^z,
\end{align}
for $0\le x,z \le N-1$ and $\varphi(x,z)$ is to be determined.

If we want $A(0,0) = d^{-1}\sum_{x,z} D(x,z)$ to be Hermitian, we need $D^{\dagger}(x,z) = D(-x,-z)$ with no extra phase. This gives the constraint
\begin{align}
	\varphi(x,z) + \varphi(-x,-z) &= xz.
\end{align}
We will show in \autoref{ch4:discrete-bochner} (a paragraph before Definition \ref{def:projective-frame}) that there is always a choice of $\varphi(x,z)$ that satisfies this constraint. The standard choice is the symmetric one $\varphi(x,z) = xz/2$, resulting in the $N^2$ displacement operators
\begin{align}
D(x,z) &= \omega^{xz/2} X^x Z^z. \label{discrete-displacement-operator}
\end{align}
In every dimension, the displacement operators enjoy the properties
\begin{align}
D^{\dagger}({\bf u}) &= D({\bf -u}), \\
D({\bf u}) D({\bf v}) &= \omega^{[{\bf u},{\bf v}]/2} D({\bf u+v}) \label{discrete-weyl-heisenberg}.
\end{align}
%Only in even dimensions, they have period $2d$
%\begin{align}
%	D({\bf x}) &= \pm D({\bf y}),& x &\equiv y \mod d,
%\end{align}
%so eq. (\ref{discrete-displacement-operator}) actually defines $4d^2$ operators in even dimensions, as opposed to $d^2$ otherwise.
%{\cmc I think I would include the stuff you just commented out and would make this next sentence definitive, conveying that in odd dimensions, it is the parity, and in even it isn't.} 
%{\nd Actually by summing up the $4d^2$ operators, you can get the identity, which is the ``parity operator" in 2 dimensions. So I wouldn't say the next sentence in definitive.} 
But only in odd dimensions is it ensured that the displacement at the origin is the parity operator, as we now show.

In the standard basis $\{\ket{m}| m=0,\dots, d-1 \}$ \eqref{discrete-position-basis},
\begin{align}
	D(x,z) = \omega^{xz/2} X^x Z^z
		&= \omega^{xz/2} \sum_m \ketbra{m+x}{m} \sum_{m'} \omega^{m'z} \ketbra{m'}{m'} \nonumber \\
		&= \sum_{m=0}^{d-1} \omega^{(m+x/2)z} \ketbra{m+x}{m}.
\end{align}
Then
\begin{align}
	A(0,0) = \frac{1}{d} \sum_{x,z} D(x,z)
		&= \sum_{x,m} \left( \frac{1}{d} \sum_z \omega^{(m+x/2)z} \right) \ketbra{m+x}{m}
\end{align}
$2^{-1}$ must be in $\mathbb{Z}_d$ for the sum to be a discrete delta function. However, the multiplicative inverse of 2 only exists for odd $d$: $2^{-1} = (d+1)/2$. Consequently, the even-dimensional displacement operators are not in the qudit Pauli group as defined. For example, the qubit Pauli group does not contain $Y = \omega^{1/2}XZ$.

In odd dimensions, we have
\begin{align}
	A(0,0) = \frac{1}{d} \sum_{x,z} D(x,z)
		&= \sum_{x,m} \left( \frac{1}{d} \sum_z \omega^{(m+x/2)z} \right) \ketbra{m+x}{m} \nonumber \\
		&= \sum_{x,m} \delta_{x,-2m} \ketbra{m+x}{m} \nonumber  \\
		&= \sum_{m=0}^{d-1} \ketbra{-m}{m}.
\end{align}
Similarly it can be shown that in the discrete Fourier basis,
\begin{align}
	A(0,0) &= \sum_{k=0}^{d-1} \ketbra{k}{-k}.
\end{align}
Furthermore, every phase point operator can be obtained from $A(0,0)$:
\begin{align}
A({\bf u}) &= D^{\dagger}({\bf u}) A(0,0) D({\bf u})
= \frac{1}{d} \sum_{\bf v} \omega^{[{\bf u}, {\bf v}]} D^{\dagger}({\bf v}), \label{wigner-covariance}
\end{align}
exactly as in infinite dimensions. (In dimension two, for example, the parity operator is the identity $\id$, which clearly cannot generate nontrivial phase point operators by conjugation.) The frame $\{A(q,p)/d\}$ defines Gross's discrete Wigner function, which can be put in the form \cite{gross_hudsons_2006,gross_non-negative_2007}
\begin{align}\label{ch4:gross-wigner-function}
\mu_{\rho}(q,p) &= \frac{1}{d} \sum_{y=0}^{d-1} \omega^{-py} \bigg\langle q+\frac{y}{2} \bigg| \rho \bigg|q-\frac{y}{2}\bigg\rangle.
\end{align}
perfectly mimicking the continuous Wigner function. (See \cite{miquel_quantum_2002} and references therein for an exposition of the even-dimensional Wigner function.)

The phase point operators in odd dimensions are very special. Since all $d^2$ displaced parity operators only have eigenvalues $\pm 1$, they are not only Hermitian but also unitary like the Pauli matrices. Moreover,
\begin{align}
A(0,0)XA(0,0) &= -X, & A(0,0)ZA(0,0) &= -Z.
\end{align}
means that they are elements of the \emph{generalized Clifford group} (not to be confused with the Clifford algebra of fermionic operators in Chapter \ref{ch:matchgate}),
the normalizer of the generalized Pauli matrices \cite{zhu_permutation_2016,bengtsson_discrete_2017}, a fact well known to those who know it well\footnote{A homage to Ivan H.~Deutsch.}, but which should be better known to those who don't. The Clifford group modulo the phases is isomorphic to the affine symplectic group $\gr{Sp}{\mathbb{Z}_d^{2n}} \ltimes \mathbb{Z}_d^{2n}$, where the \emph{semidirect product} $G = K \ltimes H$ means that $H$ is a normal subgroup of $G$---sometimes written as $G \triangleright H$, hence the semidirect product symbol $\ltimes H$---and every element of $G$ can be written uniquely as a product $kh$ (or $h'k'$) where $k,k' \in K$ and $h,h' \in H$ and $k_1h_1k_2h_2 = k_1k_2 (k_2^{-1}h_1k_2)h_2$.
We review a proof of the discrete Hudson's theorem that uses the group property of the phase point operators in Appendix \ref{app:hudson}.

%%------------------------------------
%\subsection{\nd Relation to classical simulation}\label{ch4:stabilizer}
%%------------------------------------
%
%The analogy between stabilizer and Gaussian states can be pursuit further. The \emph{stabilizer subtheory} -- the operational subtheory of quantum theory consisting of all stabilizer state preparations and generalized Pauli measurements (and implicitly Clifford transformations) -- is extremely unusual and interesting.\footnote{An ontological model ``close" to the stabilizer subtheory can be defined and studied for even-dimensional classical dits but there is no corresponding subtheory of quantum theory \cite{spekkens_evidence_2007,pusey_stabilizer_2012,catani_spekkens_2017}.} On the one hand, it displays features that many would deem exclusively quantum such as superposition and teleportation. It contains highly entangled states such as the GHZ state $d^{-n/2} \sum_{m=0}^{d-1} \ket{m}^{\otimes n}$. On the other hand, the subtheory is classical in both of the following senses:
%\begin{enumerate}
%	
%	\item its states and probabilities of obtaining a given measurement outcome are efficiently classically simulatable by the \emph{Gottesman-Knill theorem} \cite{nielsen2000quantum,aaronson_improved_2004}.
%	
%	\item By the discrete Hudson's theorem, it has as an ontological model the discrete Wigner phase space.
%\end{enumerate}
%One can readily compare the stabilizer subtheory to the classically efficiently simulatable \emph{Gaussian subtheory} -- the operational subtheory of quantum theory consisting of preparations of Gaussian states and Gaussian measurements (such as homodyne and heterodyne measurements) \cite{bartlett_efficient_2002}.

%------------------------------------
\section{The quantum Bochner's theorem}\label{ch4:bochner}
%------------------------------------

We come back to the view that the Wigner quasi-probability representation can be regarded as an independent formulation of quantum theory \cite{moyal_quantum_1949,zachos2005quantum}. To do so, it is necessary to have a criterion to distinguish valid quasi-probability distributions from invalid ones, i.e. those that do not correspond to positive operators in the Hilbert space formulation. This is accomplished by the \emph{quantum Bochner's theorem}. The positivity of the Wigner function of mixed states was studied in \cite{srinivas_nonclassical_1975,brocker_mixed_1995}. Both references independently found that a theorem in classical probability attributed to Bochner \cite{bochner_monotone_1933} and a generalization thereof can be used to characterize both the valid Wigner functions and the subset of positive ones. Surprising at the time was that positive Wigner functions were not limited to the convex hull of Gaussian states.  This fact allows generalizations of efficient classical simulation of Gaussian quantum optics and stabilizer subtheory \cite{mari_positive_2012,veitch_negative_2012, veitch_efficient_2013}.

Let us first motivate Bochner's theorem in classical probability. A probability density is a positive real function normalized to unity. A related concept is that of a \emph{positive definite function}, for which the following is satisfied for arbitrary real numbers $\zeta_0,\dots,\zeta_{N-1}$ and complex numbers $a_0,\dots,a_{N-1}$:
\begin{equation}\label{positive-definite-function}
\sum_{k,k'=0}^{N-1} \conj a_{k}a_{k'} \phi(\zeta_{k'}-\zeta_k)\geq0,
\end{equation}
for all positive integers $N$. This can be seen as a generalization of the positivity of the discrete Fourier transform of a function in the following way. Suppose that the $\xi_k$'s are equally spaced so that we can write $\xi_{k'} - \xi_k = \xi_{k'-k}$ for every $k$ and $k'$. Then eq. (\ref{positive-definite-function}) says that for every $N$, the \emph{circulant matrix}
\begin{align}\label{characteristic-matrix}
M_N &= \sum_{k,k'=0}^{N-1} \phi(\zeta_{k'}) \ketbra{k}{k-k'}
\end{align}
is positive. This matrix \eqref{characteristic-matrix} is diagonalized by the Fourier transform eq. \eqref{Fourier transform:Z_n}
\begin{align}
U_{\text{FT}} &= \frac{1}{\sqrt{N}} \sum_{x,k=0}^{N-1} \omega^{xk} \ketbra{x}{k},
\end{align}
\begin{align*}
U_{\text{FT}} M_N U\dgg_{\text{FT}}
&= \frac{1}{N} \sum_{x,x'=0}^{N-1} \sum_{k,k'=0}^{N-1}
\phi(\zeta_{k'}) \omega^{xk} \ketbra{x}{x'} \omega^{-x'(k-k')} \\
&= \sum_{x,x'=0}^{N-1} \sum_{k'=0}^{N-1} \phi(\zeta_{k'})
\left( \frac{1}{N} \sum_{k=0}^{N-1} \omega^{(x-x')k}  \right) \omega^{x'k'} \ketbra{x}{x'} \\
&= \sum_{x=0}^{N-1} \left( \sum_{k'=0}^{N-1} \phi(\zeta_{k'}) \omega^{xk'} \right) \ketbra{x}{x} \ge 0,
\end{align*}
where we have used $N^{-1} \sum_{k=0}^{N-1} \omega^{(x-x')k} = \delta_{x,x'}$. Moreover, since $M_N$ is positive, it is Hermitian with positive determinant. For example, for $N=2$ and
\begin{align}
	M_2 &= \begin{pmatrix}
		\phi(0) & \phi(\xi_0 - \xi_1) \\
		\phi(\xi_1 - \xi_0) & \phi(0)
	\end{pmatrix},
\end{align}
$|\phi(\xi_0 - \xi_1)|$ is bounded from above by $\phi(0)$ (the normalization), which is itself positive because $M_1$ is positive. This is the crux of \cite{RS2}
\begin{theorem}\label{Bochner's theorem}
	\normalfont{(Bochner's theorem)}
	The function $\phi$ is the Fourier transform of some probability density if and only if $\phi$ is continuous, positive definite and $\phi(0)=1$ (normalization).
\end{theorem}
When it comes to quasi-probability distributions, what we want to know is whether a quasi-probability distribution, which need not be positive everywhere, corresponds to a valid density operator. Generalizing the notion of positive definite function to that of a \emph{$\gamma$-positive definite function}:
\begin{equation}\label{gamma-positive-function}
\sum_{k,k'}^N \conj a_{k}a_{k'} \phi(\zeta_{k'}-\zeta_k,\eta_{k'}-\eta_k)e^{i\gamma (\zeta_k\eta_{k'}-\zeta_{k'}\eta_k)/2}\ge 0,
\end{equation}
for all positive integers $N$ (where $\gamma=0$ recovers the original positive definiteness), the quantum Bochner's theorem for the Wigner function can be stated as the following \cite{srinivas_nonclassical_1975,brocker_mixed_1995}:
\begin{theorem}\label{wigner_bochner}
	Let $\phi_\rho$ be the characteristic function \eqref{quantum-characteristic-function} of $\rho$.  Then,
	\begin{enumerate}
		\item $\rho$ is a density operator if and only if $\phi_\rho$ is 1-positive definite.
		\item $\rho$ is a density operator with positive Wigner function if and only if $\phi_\rho$ is simultaneously 1-positive definite and 0-positive definite.
	\end{enumerate}
\end{theorem}

%------------------------------------
\section{Discrete Quantum Bochner's theorem}\label{ch4:discrete-bochner}
%------------------------------------

How much can the quantum Bochner's theorem be generalized to the discrete setting? In principle, we can define the characteristic function via the notion of Fourier transform on any finite group $G$ (\autoref{ch2:fourier-transform}). But as we will see, the main difficulty is generalizing the 1-positive definiteness to functions on non-abelian groups. With that hindsight, we define the \emph{characteristic function} to be the Fourier transform of a probability distribution on an abelian group $G$. Recall the Fourier transform on an abelian group $G$ of eq. \eqref{Fourier transform:abelian}:
\begin{align}
\ket{\tilde{g}} &= \frac{1}{\sqrt{|G|}} \sum_{\lambda \in \hat{G}} \chi_{\lambda}(g) \ket{\lambda},
\end{align}
where $\chi_{\lambda}(g)$ is an irreducible character, $\chi_{\lambda}^*(g) = \chi_{\lambda}(g^{-1})$.
A function $\phi$ on $G$ then is \emph{positive definite} if the matrix $M^C_{jj'} = \phi(j'-j)$ is positive. (The superscript $C$ stands for ``classical".) The classical Bochner's theorem generalizes to this setting without change.
\begin{theorem} \label{thm:general-bochner}
	A function $\phi$ is a characteristic function on $G$ if and only if $\phi$ is positive definite and $\phi(0)=1$.
\end{theorem}
\begin{proof}
	The normalization $\phi(0)=1$ can be verified by direct calculation. For a positive definite function,
	\begin{align*}
	\sum_{j,j'=1}\conj a_{j}a_{j'}\phi(j'-j)
	&= \frac{1}{\sqrt{|G|}}\sum_{j,j'=1}\conj a_{j}a_{j'}\sum_{g}\chi_{j'-j}(g)f(g)\\
	&= \frac{1}{\sqrt{|G|}}\sum_{j,j'=1}\conj a_{j}a_{j'}\sum_{g}\chi_{j'}(g) \conj \chi_{j}(g) f(g)\\
	&= \frac{1}{\sqrt{|G|}}\sum_{g}f(g)\left|\sum_{j=1}a_{j}\chi_{j}(g)\right|^{2}.
	\end{align*}
	If $f(g) \ge 0$ for all $g$, then $\phi$ is positive definite. Conversely,
	if $f(g)<0$ at $g'$, then we can choose $a_{j}$ so that the sum $\sum_{j=1}a_{j}\chi_{j'}(g)$
	vanishes everywhere except at $g=g'$, which is possible because $\{ \chi(j) \} $
	form a basis of functions on $G$.
\end{proof}
\noindent This theorem immediately gives us a generalization of the classical Bochner's theorem in the setting of quasi-probability representations.
\begin{theorem}\label{thm:classical-bochner-quasi-rep}
	If the frame of a quasi-probability representation is the Fourier transform
	\begin{align}
	F(j) =\frac{1}{\sqrt{|G|}}\sum_{g \in G} \chi_j(g) \tilde F(g) \label{fourier-frame}
	\end{align}
	and $\phi_{\rho}(g) = \Tr[\rho \tilde F(g)]$ is a characteristic function, then $\phi_{\rho}$
	is the Fourier transform of a positive quasi-probability representation
	of $\rho$ if and only if $\phi_{\rho}$ is positive definite and $\phi(0)=1$.
\end{theorem}

Next, we need to generalize the 1-positive definiteness so that, together with Theorem \ref{thm:classical-bochner-quasi-rep}, it gives a full generalization of the quantum Bochner's theorem. The extra phase in the definition of $\gamma$-positive definite functions eq. \eqref{gamma-positive-function} comes from the property that the displacement operators $\{ D(q,p) \}$ of the Wigner function form a projective representation.

%------------------------------------
\subsection{Projective unitary frames}
%------------------------------------

A \emph{projective representation} of a group $G$ is a homomorphism from $G$ to the projective linear group in which matrices are defined only up to a scalar multiple
\begin{align}
	U:G &\to \gr{PGL}{V}.
\end{align}
Another way of putting it is that the unitary operators $\{ \rep(g) \}$ form a representation of $G$ up to a
\emph{2-cocycle}, $\alpha(g_1,g_2):G\times G\to\mathbb{C}$,
\begin{align}
\rep(g_1)\rep(g_2) & =\alpha(g_1,g_2)\rep(g_1g_2).
\end{align}
Displacement operators and generalized Pauli matrices are examples with $\alpha({\bf x,y}) = e^{i[{\bf x,y}]/2}$ and $\alpha({\bf u,v}) = \omega^{[{\bf u,v}]/2}$ respectively (see \autoref{ch4:wigner} and \autoref{ch4:discrete-wigner}). The point is that even though a group is abelian, its projective matrix representations can be non-abelian. Two projective representations $\rep$ and $\rep'$ are \emph{projectively equivalent} if there is an isomorphism up to a scalar multiple
\begin{align}
\varphi \rep(g) & =\beta(g) \rep'(g) \varphi
\end{align}
for all $g \in G$ and $\beta(e)=1$. This translates into the condition on the 2-cocycles,
\begin{align}
\alpha(g_1,g_2) &= \frac{\beta(g_1) \beta(g_2)}{\beta(g_1g_2)} \alpha'(g_1,g_2). \label{eq:cohomologous}
\end{align}
2-cocycles that give rise to equivalent projective representations are said to be \emph{cohomologous} and belong to the same \emph{2-cohomology class} $\tilde{\alpha}$. All projective representations can be classified according to their 2-cohomology class $\tilde{\alpha}$ \cite{Karpilovsky2,Karpilovsky3}.
The set of all $\tilde \alpha$ of a group $G$ is known as the \emph{Schur multiplier} of $G$. The Schur multiplier of a finite abelian group, which necessarily has the form $G \simeq \mathbb{Z}_{m_1}\times \cdots \times \mathbb{Z}_{m_n}$ by the fundamental theorem of finitely generated abelian groups,
is \cite[p.317]{Karpilovsky2}
\begin{align}\label{schur-multiplier}
	\prod_{1\le j<k\le n} \mathbb{Z}_{\text{gcd}(m_j,m_k)}.
\end{align}
Within each 2-cohomology class, we will choose to work with phases such that $|\alpha|=1$, which can always be done. For convenience, we will also choose $\rep(g)^{-1}=\rep(g^{-1})$. To do this, first choose $\alpha(g,g^{-1})=\alpha(g^{-1},g)$ for each and every $g \in G$. Then we need to find $\beta$ such that, using eq. \eqref{eq:cohomologous}
with $\alpha'(g,g')=1$, we have
\begin{align}
\alpha(g,g^{-1}) & =\beta(g)\beta(g^{-1}).
\end{align}
It is clear that $\beta(g)$ and $\beta(g^{-1})$ can be chosen independently of the values of $\beta$ at any other element. Therefore, with this choice of $\beta$, $\rep(g)^{-1} = \rep(g^{-1})$ for all $g \in G$.

Now we can define a (non-Hermitian) frame for $\mathbb{M}(\hilb)$, the \emph{complex} vector space of all linear operators on $\hilb$.

\begin{definition}\label{def:projective-frame} A \emph{projective unitary frame} $\tilde{\rep}$ is a frame for $\mathbb{M}(\hilb)$ which is also the image of a projective representation of an abelian group $G$ with $\tilde{\rep}(g)^{-1}=\tilde{\rep}(g^{-1})$.
\end{definition}

The Fourier transform of $\tilde{\rep}$ is not only a frame because the Fourier transform is an isomorphism, but also Hermitian:
\begin{align}
F(j)\dgg & =\frac{1}{|G|}\sum_{g \in G}\chi_j^*(g) \tilde{\rep}(g)\dgg = \frac{1}{|G|} \sum_{g \in G} \chi_{j}(g^{-1}) \tilde{\rep}(g^{-1}) = F(j).
\end{align}
In other words, the Fourier transform of a projective unitary frame always gives a quasi-probability representation. This is not the case if $G$ is non-abelian since Hermitian conjugation sends a matrix element $\rep(g)$ not to the same element of $\rep(g^{-1})$, but to its transpose. This precludes the non-abelian analogue of our main theorem below.

Finally, after a lot of setup, the following definition and Theorem \ref{thm:classical-bochner-quasi-rep} lead to our main theorem, Theorem \ref{thm:main-bochner}.
\begin{definition}\label{def:quantum-bochner}
	Let $G$ be a finite group and $g_1,g_2 \in G$. Given a 2-cocyle $\alpha:G\times G\to\mathbb{C}$, a function $\phi$ on $G$ is $\alpha$-positive definite if and only if the matrix defined by
	\begin{align}
	M^Q_{g_1g_2} & =\phi ( g_2g_1^{-1}) \alpha ( g_2,g_1^{-1})
	\end{align}
	is positive.
\end{definition}
\begin{theorem}\label{thm:main-bochner}
	If the frame of a quasi-probability representation is the Fourier transform of a projective unitary frame $\tilde{F}$ of a finite abelian group $G$ with $\alpha(g_1^{-1},g_2)$ being a 2-cocycle and $\phi_{\rho}(g) = \Tr[\rho \tilde{F}(g)]$ is a characteristic function, then the followings hold:
	\begin{enumerate}
		\item $\rho$ is a density operator if and only if $\phi_{\rho}$ is $\alpha$-positive definite,
		\item $\rho$ is a density operator with positive quasi-probability representation if and only
		if $\phi_{\rho}$ is simultaneously $\alpha$-positive definite and positive definite.
	\end{enumerate}
\end{theorem}	
\begin{proof}
	An arbitrary operator $A$ can be expanded using the projective unitary frame. Since $A\dgg A$ is a positive operator,
	\begin{align*}
	\Tr (\rho A\dgg A)
	&= \sum_{g_1,g_2}\conj a(g_1)a(g_2) \Tr [\rho \tilde{\rep}(g_1)\dgg  \tilde{\rep}(g_2)] \\
	&= \sum_{g_1,g_2}\conj a(g_1)a(g_2) \Tr [\rho \tilde{\rep}(g_1^{-1})  \tilde{\rep}(g_2)] \\
	&= \sum_{g_1,g_2}\conj a(g_1)a(g_2) \Tr [\rho \tilde{\rep}(g_1^{-1}g_2)] \alpha(g_1^{-1},g_2)\\
	&= \sum_{g_1,g_2}\conj a(g_1)a(g_2) \phi_{\rho} (g_2g_1^{-1}) \alpha(g_1^{-1},g_2) \ge 0
	\end{align*}
	if and only if $\rho$ is a positive operator. Theorem \ref{thm:classical-bochner-quasi-rep} then completes the proof.
\end{proof}


%------------------------------------
\subsection{Characterizing projective unitary frames}
%------------------------------------

Since the generalized Pauli matrices in \autoref{ch4:discrete-wigner} form a projective frame, this gives the quantum Bochner's theorem for discrete Wigner function in all dimensions, odd or even. The discrete Wigner functions in odd dimensions are examples of faithful projective representations---called \emph{faithful projective frames}---and the ones in even dimensions are examples of those that are not faithful. %The latter class can provide overcomplete frames for constructing quasi-probability representations, but not always.
We have the theorem:
\begin{theorem}
	\begin{enumerate}
		\item A faithful projective frame $\tilde{\rep}$ exists if and only if $G$
		is a symmetric product of groups $H \times H$ with $|G| = d^2$.
		\item A faithful projective frame $\tilde{\rep}$ and its Fourier frame are both orthogonal frames---that is, orthogonal bases.
	\end{enumerate}
\end{theorem}
\begin{proof}
	1. Being a frame forces a projective frame to be irreducible as a projective representation. It is known that $G=H \times H$ with $|G|= d^2$ if and only if it has a faithful irreducible projective representation \cite[Theorem 8.2.18]{Karpilovsky3}. It is worth noting that this implies faithful projective frames are precisely the \emph{nice error bases} of quantum error correction with abelian index group \cite{knill_group_1996}.
	
    2. We first prove that every $\tilde{\rep}(g)$, $g\neq e$, is traceless by writing it as a commutator. Fixing $g\neq e$, $\tilde{\rep}(g)$ is not proportional to the identity operator by our assumption of faithfulness, so $g'$ can be found such that $[\tilde{\rep}(g),\tilde{\rep}(g')] \neq 0$ because otherwise $\tilde{\rep}$ is reducible (for $d>1$). By the group property, we can write $\tilde{\rep}_{g}$ as a product of two non-commuting operators,
	\begin{align}
	\alpha(g',g'^{-1}g) \tilde{\rep}(g) &= \tilde{\rep}(g') \tilde{\rep}(g'^{-1}g).
	\end{align}
	But $G$ is abelian, so
	\begin{align}
	0 \neq [ \tilde{\rep}(g'), \tilde{\rep}(g'^{-1}g) ] &= [\alpha (g',g'^{-1}g) - \alpha (g'^{-1}g,g') ] \tilde{\rep}(g).
	\end{align}
	Take the trace of both sides and the claim is proved.
	
	Consequently, $\{ \tilde{\rep}(g) \}$ are all orthogonal in the trace inner product.
	Therefore, since $|\{\tilde{\rep}\}|=d^2$, it is an orthogonal basis of $\mathbb{M}(\hilb)$, as its Fourier transform is an orthogonal basis of $\mathbb{H}(\hilb)$.
\end{proof}


The generating matrices of every representative faithful projective frame up
to $d=7$ are listed at \cite{klappenecker_beyond_2002}.
For general $d$, as long as we only consider representations over $\mathbb{C}$, at least one representative projective representation of each and every 2-cohomology classes of $G$ appears as an ordinary representation of a (non-unique) \emph{covering group} of $G$, which can be found, for instance, by the command \texttt{SchurCover(G)}
in \texttt{GAP} \cite{GAP4}.


To deal with overcomplete frames, we will need to know about the kernel. The kernel of a projective representation is a set of group elements that are mapped to the identity operator up to a phase. By the first isomorphism theorem, ker $\rep$ is a normal subgroup of $G$ and a (projective) representation is faithful if and only if ker $\rep=\{ e \} $. Therefore, any $\rep$ descends to a faithful projective representation $\upsilon$ of $G/\ker\rep$ defined by $\upsilon\left(g\ker\rep\right)=\rep(g)$ up to a phase and vice versa.
\begin{align}
\xymatrix{G\ar[r]^{\rep}\ar[d] & \gr{PGL}{\hilb} \\
	G/\ker\rep \ar[ur]_{\displaystyle \upsilon} & }
\end{align}
Thus there is a one-one correspondence between projective representations of $G$ and of $G/\ker\rep$, which preserves irreducibility and the property of $\{ U(g) \}$ being a frame. As a result, the task of finding an unfaithful projective frame reduces to the task of lifting a faithful projective frame of an abelian group $H \times H$ with $ |H|= d$ to the corresponding projective frame of a group $G$ which has $\ker\tilde{\rep}$ as a normal subgroup and $G/\ker\tilde{\rep} = H \times H$. Finding such $G$ is an abelian \emph{extension problem}. Note that $\ker\tilde{\rep}$ can be any abelian group.

To summarize, the procedure to construct a quasi-probability representation with the quantum Bochner's theorem is as follows:
\begin{enumerate}
	\item Pick an abelian group $H$ of size $d$, the dimension of the quantum system.
	\item Extend the group $H \times H$ by a (possibly trivial) abelian group to $G$.
	\item Construct an irreducible projective representation $\{\rep\}$ of $G$ up to projective equivalence. Within the equivalence class, each unitary operator is still only defined up to a phase. Choose the phases under the constraint $|\alpha|=1$ and $\rep(g)^{-1}=\rep(g^{-1})$ for all $g\in G$. The set of operators is now a projective frame.
	\item Fourier transform the projective frame according to eq. \eqref{fourier-frame} to obtain the frame of the quasi-probability representation.
\end{enumerate}

%------------------------------------
\subsection{Examples and non-examples}
%------------------------------------

Let us illustrate the kind of quasi-probability representations that can arise in the characterization of faithful projective frames by examples in $d=2,3,4$. We know how many cohomology classes there are from their Schur multipliers of eq.~\eqref{schur-multiplier}. For $d=2$, there is only $G=\mathbb{Z}_{2}^{2}$ and one 2-cohomology class with the Pauli matrices $\{ \id,X,Y,Z \} $ as a representative projective frame. The requirement that $\tilde{F}(g)^{-1}=\tilde{F}(g^{-1})$ constrains $\tilde{F}$ to be Hermitian in addition to being unitary, leaving us with the choices to put $\pm1$ in front of $X,Y,$ or $Z$. But since $\chi_{j}(g)=\pm1$ also, upon doing the Fourier transform, we end up with only two quasi-probability representations depending on whether we change the phases of an odd or even number of Pauli matrices. They are related not by a unitary transformation but by an anti-unitary transformation. They coincide with the two similarity classes of the qubit phase space identified in \cite{gibbons_discrete_2004}. For $d=3$, there is only $G=\mathbb{Z}_3^2$, and two inequivalent projective representations but with the same image. The phase freedom that remains after setting $\tilde{F}(g)^{-1}=\tilde{F}(g^{-1})$ is enough to make the set of quasi-probability representations generated from the two classes identical. The phase freedom, however, supplies a continuum of choices in choosing different $\tilde{F}$, whose quasi-probability representations are in general not unitarily related, one of them being the discrete Wigner function of Gross. The case $d=4$ is the first instance with more than one choice of group ($\mathbb{Z}_4^2$ or $\mathbb{Z}_2^4$) and inequivalent projective representations (of $\mathbb{Z}_2^4$) that generate distinct quasi-probability representations.

An unfaithful projective frame of size $N$ can give rise to a frame of size less than $N$. As an example, suppose that the quantum system is 2 dimensional and the group is $\mathbb{Z}_2^3$. There is an irreducible projective representation sending the elements (0,0,0),(1,0,0) to $\id$, (0,0,1),(1,0,1) to $X$, (0,1,0),(1,1,0) to $Z$, and (0,1,1),(1,1,1) to $Y$. (That is, (1,0,0) is in the kernel of this representation.) But one can find an irreducible character sending the first group element of each pair above to 1 while sending the other element to -1. This component of the Fourier transform is therefore zero, which is not surprising because this projective frame is highly redundant. (All the nonzero components still form a basis.)

There are many discrete analogues of the Wigner function identified in \cite{ferrie_quasi-probability_2011} which do not possess the symmetry we have assumed here, for example, Hardy's vector
representation \cite{hardy_quantum_2001} and the representation based on symmetric informationally complete (SIC) POVMs \cite{zauner_quantum_2011,renes_symmetric_2004} whose frame is a set of rank-one projection operators that form a basis. There is no projective frame for this representation since a complete set of projection operators cannot all be pairwise orthogonal. 